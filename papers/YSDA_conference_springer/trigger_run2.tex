% This is LLNCS.DEM the demonstration file of
% the LaTeX macro package from Springer-Verlag
% for Lecture Notes in Computer Science,
% version 2.4 for LaTeX2e as of 16. April 2010
%
\documentclass{llncs}
%
\usepackage{makeidx}  % allows for indexgeneration
\usepackage{hyperref}
\usepackage{graphicx}
\usepackage{caption}
\usepackage{subcaption}
%
\begin{document}
%
\title{LHCb Topological Trigger: Approaches and Optimization}

\titlerunning{LHCb Topological Trigger}  % abbreviated title (for running head)
%                                     also used for the TOC unless
%                                     \toctitle is used
%
\author{Tatiana Likhomanenko\inst{1, 2, 3} \email{tatiana.likhomanenko@cern.ch} \and Philip Ilten\inst{5} \and Egor Khairullin\inst{1, 4} \and Alex Rogozhnikov\inst{1, 2} \and Andrey Ustyuzhanin\inst{1, 2, 3, 4} \and Michael Williams\inst{5} \email{mwill@mit.edu}}
\authorrunning{Ivar Ekeland et al.} % abbreviated author list (for running head)
%
%%%% list of authors for the TOC (use if author list has to be modified)
% \tocauthor{Ivar Ekeland, Roger Temam, Jeffrey Dean, David Grove,
% Craig Chambers, Kim B. Bruce, and Elisa Bertino}
% %
% \institute{Princeton University, Princeton NJ 08544, USA,\\
% \email{I.Ekeland@princeton.edu},\\ WWW home page:
% \texttt{http://users/\homedir iekeland/web/welcome.html}
% \and
% Universit\'{e} de Paris-Sud,
% Laboratoire d'Analyse Num\'{e}rique, B\^{a}timent 425,\\
% F-91405 Orsay Cedex, France}

\institute{ Yandex School of Data Analysis (YSDA), RU
\and
National Research University Higher School of Economics (HSE), RU
\and 
NRC "Kurchatov Institute", RU
\and 
Moscow Institute of Physics and Technology, Moscow, RU
\and 
Massachusetts Institute of Technology, US}



\maketitle              % typeset the title of the contribution

\begin{abstract}
The LHCb trigger system plays a key role in selecting events (proton-proton collisions) to store them into memory for further processing. It should choose several thousands events per second among all detected by LHCb. The main b-physics trigger algorithm (should select events with B-meson decays) used by the LHCb experiment is a so-called topological trigger. In the LHC Run 1, this trigger, which utilized a simple boosted decision tree algorithm, selected a nearly 100\% pure sample of b-hadrons with a typical efficiency of 60-70\%; its output was used in about 60\% of LHCb papers. In the paper \cite{run2_topo} we presented studies carried out to optimize the topological trigger for LHC Run 2. Trigger data have a specific structure and that is why necessary quality measure was proposed. In this paper we investigate the presense of noisy samples in data and propose several ways to reduce this noise. Anoter side of the trigger system is a real-time prediction operation: it should be very fast. In the paper \cite{run2_topo} two approaches how to speedup a prediction operation and to preserve the quality as possible were considered: bonsai boosted decision tree format (used in Run 1) and decision trees post prunning. In this paper we analyse behaviour of the cleaned up samples for these two approaches. As a result, we demonstrate that removing of noise can still improve reoptimized topological trigger on the Run 1 performance for a wide range of b-hadron decays.

\keywords{high energy physics, machine learning, LHCb trigger system}
\end{abstract}
%
% \section{Fixed-Period Problems: The Sublinear Case}
% %
% With this chapter, the preliminaries are over, and we begin the search
% for periodic solutions to Hamiltonian systems. All this will be done in
% the convex case; that is, we shall study the boundary-value problem
% \begin{eqnarray*}
%   \dot{x}&=&JH' (t,x)\\
%   x(0) &=& x(T)
% \end{eqnarray*}
% with $H(t,\cdot)$ a convex function of $x$, going to $+\infty$ when
% $\left\|x\right\| \to \infty$.

% %
% \subsection{Autonomous Systems}
% %
% In this section, we will consider the case when the Hamiltonian $H(x)$
% is autonomous. For the sake of simplicity, we shall also assume that it
% is $C^{1}$.

% We shall first consider the question of nontriviality, within the
% general framework of
% $\left(A_{\infty},B_{\infty}\right)$-subquadratic Hamiltonians. In
% the second subsection, we shall look into the special case when $H$ is
% $\left(0,b_{\infty}\right)$-subquadratic,
% and we shall try to derive additional information.
% %
% \subsubsection{The General Case: Nontriviality.}
% %
% We assume that $H$ is
% $\left(A_{\infty},B_{\infty}\right)$-sub\-qua\-dra\-tic at infinity,
% for some constant symmetric matrices $A_{\infty}$ and $B_{\infty}$,
% with $B_{\infty}-A_{\infty}$ positive definite. Set:
% \begin{eqnarray}
% \gamma :&=&{\rm smallest\ eigenvalue\ of}\ \ B_{\infty} - A_{\infty} \\
%   \lambda : &=& {\rm largest\ negative\ eigenvalue\ of}\ \
%   J \frac{d}{dt} +A_{\infty}\ .
% \end{eqnarray}

% Theorem~\ref{ghou:pre} tells us that if $\lambda +\gamma < 0$, the
% boundary-value problem:
% \begin{equation}
% \begin{array}{rcl}
%   \dot{x}&=&JH' (x)\\
%   x(0)&=&x (T)
% \end{array}
% \end{equation}
% has at least one solution
% $\overline{x}$, which is found by minimizing the dual
% action functional:
% \begin{equation}
%   \psi (u) = \int_{o}^{T} \left[\frac{1}{2}
%   \left(\Lambda_{o}^{-1} u,u\right) + N^{\ast} (-u)\right] dt
% \end{equation}
% on the range of $\Lambda$, which is a subspace $R (\Lambda)_{L}^{2}$
% with finite codimension. Here
% \begin{equation}
%   N(x) := H(x) - \frac{1}{2} \left(A_{\infty} x,x\right)
% \end{equation}
% is a convex function, and
% \begin{equation}
%   N(x) \le \frac{1}{2}
%   \left(\left(B_{\infty} - A_{\infty}\right) x,x\right)
%   + c\ \ \ \forall x\ .
% \end{equation}

% %
% \begin{proposition}
% Assume $H'(0)=0$ and $ H(0)=0$. Set:
% \begin{equation}
%   \delta := \liminf_{x\to 0} 2 N (x) \left\|x\right\|^{-2}\ .
%   \label{eq:one}
% \end{equation}

% If $\gamma < - \lambda < \delta$,
% the solution $\overline{u}$ is non-zero:
% \begin{equation}
%   \overline{x} (t) \ne 0\ \ \ \forall t\ .
% \end{equation}
% \end{proposition}
% %
% \begin{proof}
% Condition (\ref{eq:one}) means that, for every
% $\delta ' > \delta$, there is some $\varepsilon > 0$ such that
% \begin{equation}
%   \left\|x\right\| \le \varepsilon \Rightarrow N (x) \le
%   \frac{\delta '}{2} \left\|x\right\|^{2}\ .
% \end{equation}

% It is an exercise in convex analysis, into which we shall not go, to
% show that this implies that there is an $\eta > 0$ such that
% \begin{equation}
%   f\left\|x\right\| \le \eta
%   \Rightarrow N^{\ast} (y) \le \frac{1}{2\delta '}
%   \left\|y\right\|^{2}\ .
%   \label{eq:two}
% \end{equation}

% \begin{figure}
% \vspace{2.5cm}
% \caption{This is the caption of the figure displaying a white eagle and
% a white horse on a snow field}
% \end{figure}

% Since $u_{1}$ is a smooth function, we will have
% $\left\|hu_{1}\right\|_\infty \le \eta$
% for $h$ small enough, and inequality (\ref{eq:two}) will hold,
% yielding thereby:
% \begin{equation}
%   \psi (hu_{1}) \le \frac{h^{2}}{2}
%   \frac{1}{\lambda} \left\|u_{1} \right\|_{2}^{2} + \frac{h^{2}}{2}
%   \frac{1}{\delta '} \left\|u_{1}\right\|^{2}\ .
% \end{equation}

% If we choose $\delta '$ close enough to $\delta$, the quantity
% $\left(\frac{1}{\lambda} + \frac{1}{\delta '}\right)$
% will be negative, and we end up with
% \begin{equation}
%   \psi (hu_{1}) < 0\ \ \ \ \ {\rm for}\ \ h\ne 0\ \ {\rm small}\ .
% \end{equation}

% On the other hand, we check directly that $\psi (0) = 0$. This shows
% that 0 cannot be a minimizer of $\psi$, not even a local one.
% So $\overline{u} \ne 0$ and
% $\overline{u} \ne \Lambda_{o}^{-1} (0) = 0$. \qed
% \end{proof}
% %
% \begin{corollary}
% Assume $H$ is $C^{2}$ and
% $\left(a_{\infty},b_{\infty}\right)$-subquadratic at infinity. Let
% $\xi_{1},\allowbreak\dots,\allowbreak\xi_{N}$  be the
% equilibria, that is, the solutions of $H' (\xi ) = 0$.
% Denote by $\omega_{k}$
% the smallest eigenvalue of $H'' \left(\xi_{k}\right)$, and set:
% \begin{equation}
%   \omega : = {\rm Min\,} \left\{\omega_{1},\dots,\omega_{k}\right\}\ .
% \end{equation}
% If:
% \begin{equation}
%   \frac{T}{2\pi} b_{\infty} <
%   - E \left[- \frac{T}{2\pi}a_{\infty}\right] <
%   \frac{T}{2\pi}\omega
%   \label{eq:three}
% \end{equation}
% then minimization of $\psi$ yields a non-constant $T$-periodic solution
% $\overline{x}$.
% \end{corollary}
% %

% We recall once more that by the integer part $E [\alpha ]$ of
% $\alpha \in \bbbr$, we mean the $a\in \bbbz$
% such that $a< \alpha \le a+1$. For instance,
% if we take $a_{\infty} = 0$, Corollary 2 tells
% us that $\overline{x}$ exists and is
% non-constant provided that:

% \begin{equation}
%   \frac{T}{2\pi} b_{\infty} < 1 < \frac{T}{2\pi}
% \end{equation}
% or
% \begin{equation}
%   T\in \left(\frac{2\pi}{\omega},\frac{2\pi}{b_{\infty}}\right)\ .
%   \label{eq:four}
% \end{equation}

% %
% \begin{proof}
% The spectrum of $\Lambda$ is $\frac{2\pi}{T} \bbbz +a_{\infty}$. The
% largest negative eigenvalue $\lambda$ is given by
% $\frac{2\pi}{T}k_{o} +a_{\infty}$,
% where
% \begin{equation}
%   \frac{2\pi}{T}k_{o} + a_{\infty} < 0
%   \le \frac{2\pi}{T} (k_{o} +1) + a_{\infty}\ .
% \end{equation}
% Hence:
% \begin{equation}
%   k_{o} = E \left[- \frac{T}{2\pi} a_{\infty}\right] \ .
% \end{equation}

% The condition $\gamma < -\lambda < \delta$ now becomes:
% \begin{equation}
%   b_{\infty} - a_{\infty} <
%   - \frac{2\pi}{T} k_{o} -a_{\infty} < \omega -a_{\infty}
% \end{equation}
% which is precisely condition (\ref{eq:three}).\qed
% \end{proof}
% %

% \begin{lemma}
% Assume that $H$ is $C^{2}$ on $\bbbr^{2n} \setminus \{ 0\}$ and
% that $H'' (x)$ is non-de\-gen\-er\-ate for any $x\ne 0$. Then any local
% minimizer $\widetilde{x}$ of $\psi$ has minimal period $T$.
% \end{lemma}
% %
% \begin{proof}
% We know that $\widetilde{x}$, or
% $\widetilde{x} + \xi$ for some constant $\xi
% \in \bbbr^{2n}$, is a $T$-periodic solution of the Hamiltonian system:
% \begin{equation}
%   \dot{x} = JH' (x)\ .
% \end{equation}

% There is no loss of generality in taking $\xi = 0$. So
% $\psi (x) \ge \psi (\widetilde{x} )$
% for all $\widetilde{x}$ in some neighbourhood of $x$ in
% $W^{1,2} \left(\bbbr / T\bbbz ; \bbbr^{2n}\right)$.

% But this index is precisely the index
% $i_{T} (\widetilde{x} )$ of the $T$-periodic
% solution $\widetilde{x}$ over the interval
% $(0,T)$, as defined in Sect.~2.6. So
% \begin{equation}
%   i_{T} (\widetilde{x} ) = 0\ .
%   \label{eq:five}
% \end{equation}

% Now if $\widetilde{x}$ has a lower period, $T/k$ say,
% we would have, by Corollary 31:
% \begin{equation}
%   i_{T} (\widetilde{x} ) =
%   i_{kT/k}(\widetilde{x} ) \ge
%   ki_{T/k} (\widetilde{x} ) + k-1 \ge k-1 \ge 1\ .
% \end{equation}

% This would contradict (\ref{eq:five}), and thus cannot happen.\qed
% \end{proof}
% %
% \paragraph{Notes and Comments.}
% The results in this section are a
% refined version of \cite{clar:eke};
% the minimality result of Proposition
% 14 was the first of its kind.

% To understand the nontriviality conditions, such as the one in formula
% (\ref{eq:four}), one may think of a one-parameter family
% $x_{T}$, $T\in \left(2\pi\omega^{-1}, 2\pi b_{\infty}^{-1}\right)$
% of periodic solutions, $x_{T} (0) = x_{T} (T)$,
% with $x_{T}$ going away to infinity when $T\to 2\pi \omega^{-1}$,
% which is the period of the linearized system at 0.

% \begin{table}
% \caption{This is the example table taken out of {\it The
% \TeX{}book,} p.\,246}
% \begin{center}
% \begin{tabular}{r@{\quad}rl}
% \hline
% \multicolumn{1}{l}{\rule{0pt}{12pt}
%                    Year}&\multicolumn{2}{l}{World population}\\[2pt]
% \hline\rule{0pt}{12pt}
% 8000 B.C.  &     5,000,000& \\
%   50 A.D.  &   200,000,000& \\
% 1650 A.D.  &   500,000,000& \\
% 1945 A.D.  & 2,300,000,000& \\
% 1980 A.D.  & 4,400,000,000& \\[2pt]
% \hline
% \end{tabular}
% \end{center}
% \end{table}
% %
% \begin{theorem} [Ghoussoub-Preiss]\label{ghou:pre}
% Assume $H(t,x)$ is
% $(0,\varepsilon )$-subquadratic at
% infinity for all $\varepsilon > 0$, and $T$-periodic in $t$
% \begin{equation}
%   H (t,\cdot )\ \ \ \ \ {\rm is\ convex}\ \ \forall t
% \end{equation}
% \begin{equation}
%   H (\cdot ,x)\ \ \ \ \ {\rm is}\ \ T{\rm -periodic}\ \ \forall x
% \end{equation}
% \begin{equation}
%   H (t,x)\ge n\left(\left\|x\right\|\right)\ \ \ \ \
%   {\rm with}\ \ n (s)s^{-1}\to \infty\ \ {\rm as}\ \ s\to \infty
% \end{equation}
% \begin{equation}
%   \forall \varepsilon > 0\ ,\ \ \ \exists c\ :\
%   H(t,x) \le \frac{\varepsilon}{2}\left\|x\right\|^{2} + c\ .
% \end{equation}

% Assume also that $H$ is $C^{2}$, and $H'' (t,x)$ is positive definite
% everywhere. Then there is a sequence $x_{k}$, $k\in \bbbn$, of
% $kT$-periodic solutions of the system
% \begin{equation}
%   \dot{x} = JH' (t,x)
% \end{equation}
% such that, for every $k\in \bbbn$, there is some $p_{o}\in\bbbn$ with:
% \begin{equation}
%   p\ge p_{o}\Rightarrow x_{pk} \ne x_{k}\ .
% \end{equation}
% \qed
% \end{theorem}
% %
% \begin{example} [{{\rm External forcing}}]
% Consider the system:
% \begin{equation}
%   \dot{x} = JH' (x) + f(t)
% \end{equation}
% where the Hamiltonian $H$ is
% $\left(0,b_{\infty}\right)$-subquadratic, and the
% forcing term is a distribution on the circle:
% \begin{equation}
%   f = \frac{d}{dt} F + f_{o}\ \ \ \ \
%   {\rm with}\ \ F\in L^{2} \left(\bbbr / T\bbbz; \bbbr^{2n}\right)\ ,
% \end{equation}
% where $f_{o} : = T^{-1}\int_{o}^{T} f (t) dt$. For instance,
% \begin{equation}
%   f (t) = \sum_{k\in \bbbn} \delta_{k} \xi\ ,
% \end{equation}
% where $\delta_{k}$ is the Dirac mass at $t= k$ and
% $\xi \in \bbbr^{2n}$ is a
% constant, fits the prescription. This means that the system
% $\dot{x} = JH' (x)$ is being excited by a
% series of identical shocks at interval $T$.
% \end{example}
% %
% \begin{definition}
% Let $A_{\infty} (t)$ and $B_{\infty} (t)$ be symmetric
% operators in $\bbbr^{2n}$, depending continuously on
% $t\in [0,T]$, such that
% $A_{\infty} (t) \le B_{\infty} (t)$ for all $t$.

% A Borelian function
% $H: [0,T]\times \bbbr^{2n} \to \bbbr$
% is called
% $\left(A_{\infty} ,B_{\infty}\right)$-{\it subquadratic at infinity}
% if there exists a function $N(t,x)$ such that:
% \begin{equation}
%   H (t,x) = \frac{1}{2} \left(A_{\infty} (t) x,x\right) + N(t,x)
% \end{equation}
% \begin{equation}
%   \forall t\ ,\ \ \ N(t,x)\ \ \ \ \
%   {\rm is\ convex\ with\  respect\  to}\ \ x
% \end{equation}
% \begin{equation}
%   N(t,x) \ge n\left(\left\|x\right\|\right)\ \ \ \ \
%   {\rm with}\ \ n(s)s^{-1}\to +\infty\ \ {\rm as}\ \ s\to +\infty
% \end{equation}
% \begin{equation}
%   \exists c\in \bbbr\ :\ \ \ H (t,x) \le
%   \frac{1}{2} \left(B_{\infty} (t) x,x\right) + c\ \ \ \forall x\ .
% \end{equation}

% If $A_{\infty} (t) = a_{\infty} I$ and
% $B_{\infty} (t) = b_{\infty} I$, with
% $a_{\infty} \le b_{\infty} \in \bbbr$,
% we shall say that $H$ is
% $\left(a_{\infty},b_{\infty}\right)$-subquadratic
% at infinity. As an example, the function
% $\left\|x\right\|^{\alpha}$, with
% $1\le \alpha < 2$, is $(0,\varepsilon )$-subquadratic at infinity
% for every $\varepsilon > 0$. Similarly, the Hamiltonian
% \begin{equation}
% H (t,x) = \frac{1}{2} k \left\|k\right\|^{2} +\left\|x\right\|^{\alpha}
% \end{equation}
% is $(k,k+\varepsilon )$-subquadratic for every $\varepsilon > 0$.
% Note that, if $k<0$, it is not convex.
% \end{definition}
% %


\section{Introduction}
The LHCb detector \cite{detector} is designed for studying beauty and charm (heavy flavour) hadrons produced in proton-proton collisions at the Large Hadron Collider (LHC) at CERN.
At LHCb data were collected with the rate 40 MHz in Run 1 ($40*10^6$ proton-proton collisions, called events, per second). This amount cannot be store into memory. A trigger system hardware part, called Level-0 (L0), reduced the visible bunch crossing rate to 1 MHz at which the detector could be read out. Further, a flexible software High-Level Trigger (HLT) applied a range of more advanced selections that reduce the rate to about 5 kHz ($5 * 10^3$ events per second) for offline storage and processing. This configuration allowed LHCb to record the largest beauty and charm hadron samples at a very high signal purity. The performance of the Run 1 trigger is described in detail in \cite{run1_1}, \cite{run1_2}.

In Run 1 it was shown that complex, multivariate trigger selections were possible. The HLT processes few enough events that it is possible to perform the reconstruction of a collision (to reconstruct tracks, vertices, a topology, physical characteristics). There are many HLT lines dedicated to triggering on various types of events. The majority of analyses using b-hadrons at LHCb made use of the topological n-body trigger \cite{run1_topo}. Most n-body hadronic B decays $(n \geq 3)$ are only triggered on efficiently in LHCb by these lines. The topological n-body trigger is an inclusive trigger which combines successive 2,3 and 4-body track combinations. A novel boosted decision trees (BDT) multivariate selection \cite{bbdt} was used in Run 1. 

In the \cite{run2_topo} we studied upgrading of the topological trigger: a new scheme of the trigger processing, which includes a more widespread usage of machine learning, was proposed; the appropriate quality measure was considered (receiver operating characteristic, ROC, computed for events for all B decays). However, obtained algorithms cannot be used in real-time trigger data processing. That is why, in the paper \cite{run2_topo} we compared two approaches how to speedup a prediction operation: bonsai boosted decision tree format \cite{bbdt} and decision trees post prunning. 

In the previous analysis \cite{run2_topo} simulated data for various B decays were used as signal-like samples and real data from the collider without any physics were used as background-like samples. In this paper we present studies connected with the fact that simulated samples contain noisy data. We propose several ways how to clean up the samples and analyse the quality of the two speedup approaches in this case. As a result, we demonstrate that removing of noise can still improve the reoptimized topological trigger on the Run 1 performance for a wide range of b-hadron decays.

\section{Motivation}
Data for the trigger system have a specific structure: each event consists of the several reconstructed secondary vertices; a set of all secondary vertices for all B decays are the training sample. After training an event will pass the trigger if at least one its secondary vertex will pass the trigger (passing the trigger means that a prediction for a sample is greater than some fixed threshold; a threshold is chosen in a way that the false positive rate value provides the trigger rate). And we are interested in the greatest true positive rate for each B-decay mode. 

As background-like data we use real data from the collider without any physics (that is why chosing the trigger rate will be equivalent to chosing the false posive rate). It means that all reconstructed secondary vertices for each event in the background-like data don't contain any interesting decays. As signal-like data we use the proton-proton collisions simulated for each mode of B-decays. For these data we really know that at least in one secondary vertex necessary decay happened. But it doesn't guaratee that all secondary vertices contain interesting B-decays.

Each event can contain up to several hundreds secondary vertices. Most of them are removed from the training data by several physical selections wchich are applyed before. In the training data each event contains up to 124 secondary vertices and the mean equals to 6. For the signal-like data one of these is a truly signal-like sample, but others can be noise.

\section{Noise clean up approaches}

Specific data structure and physical properties lead to the necessity to clean up only signal-like samples. It is well-known \cite{forest} that random forest algorithm is stable with respect to noise and outliers. That is why random forest probability-predictions for signal-like data will be greater for truly signal-like samples and lower for output noise samples (samples which should be marked as background-like). The first idea of the random forest application is the following: 
\begin{enumerate}
	\item train random forest on the full training data;
	\item choose for each signal-like event one secondary vertex which has the greatest random forest prediction (in the training data);
	\item train original algorithm on the selected training signal-like samples and all training background-like samples.
\end{enumerate}

This approach was applied to the topological trigger HLT2 (n-body, $n \geq 2$~). Training data for this line contain only six B-decays and test data contain 20 B-decays including these training modes. In figures \ref{fig:forest_simple} this approach is presented in comparison with the original algorithm without any preselections of the secondary vertices. Here also another possible way, in which for each event two secondary vertices with the greatest predictions are taken on the second step, is shown.

\begin{figure}
	\begin{center}
    	\begin{subfigure}[b]{0.45\textwidth}
    		\includegraphics[width=\textwidth]{../../img/roc_events.png} \caption{}
    	\end{subfigure} %forest_sel_top_train} 
    	\begin{subfigure}[b]{0.45\textwidth}
    		\includegraphics[width=\textwidth]{../../img/roc_events.png} \caption{} %forest_sel_top}
    	\end{subfigure}
    \end{center}
  \caption{Comparison between the original algorithm (\texttt{base}) and the random forest preselection approach: \texttt{forest selection, top 1} --- preselect for each event a secondary vertex with the greatest random forest prediction; \texttt{forest selection, top 2} --- preselect for each event two secondary vertices with the greatest random forest predictions. ROC curves is produced only for six training modes on (a) and for all available B-decays on (b).}~\label{fig:forest_simple}
\end{figure}

Because data are the mixture of several B-decays and these modes have different recognition quality there are several possible approaches to clean up the data:
\begin{itemize}
	\item Do preselection only for well recognized B-decays (in our case there are 4 well determined B-decay), Figure \ref{fig:forest_partial}:
		\begin{enumerate}
			\item train random forest on the full training data;
			\item in each well determined mode choose for each signal-like event one-two secondary vertices which have the greatest random forest predictions (in the training data); for each non-well determined modes in the trainig data take all secondary vertices;
			\item train original algorithm on the selected training signal-like samples and all training background-like samples.
		\end{enumerate}

	\item Use different random forests for preselections depending on the mode, Figure \ref{fig:forest_channel}:
	\begin{enumerate}
			\item for each mode train random forest on the training B-decay mode samples and full background-like samples;
			\item choose for each signal-like event one-two secondary vertices which have the greatest random forest (corresponding to the event's mode type) predictions (in the training data);
			\item train original algorithm on the selected training signal-like samples and all training background-like samples.
		\end{enumerate}
	\item Use different algorithms apart from random forests for preselections depending on the mode. We used XGBoost, Figure \ref{fig:xgb_channel}.
\end{itemize}

\begin{figure}
	\begin{center}
    	\begin{subfigure}[b]{0.45\textwidth}
    		\includegraphics[width=\textwidth]{../../img/roc_events.png} \caption{}
    	\end{subfigure} %forest_sel_top_train} 
    	\begin{subfigure}[b]{0.45\textwidth}
    		\includegraphics[width=\textwidth]{../../img/roc_events.png} \caption{} %forest_sel_top}
    	\end{subfigure}
    \end{center}
  \caption{Comparison between the original algorithm (\texttt{base}) and the random forest preselection approach: \texttt{forest selection partial, top 1} , \texttt{forest selection partial, top 2}. ROC curves is produced only for six training modes on (a) and for all available B-decays on (b).}~\label{fig:forest_partial}
\end{figure}


\begin{figure}
	\begin{center}
    	\begin{subfigure}[b]{0.45\textwidth}
    		\includegraphics[width=\textwidth]{../../img/roc_events.png} \caption{}
    	\end{subfigure} %forest_sel_top_train} 
    	\begin{subfigure}[b]{0.45\textwidth}
    		\includegraphics[width=\textwidth]{../../img/roc_events.png} \caption{} %forest_sel_top}
    	\end{subfigure}
    \end{center}
  \caption{Comparison between the original algorithm (\texttt{base}) and the random forest preselection approach: \texttt{forest top 1 in channels}, \texttt{forest top 2 in channels}. ROC curves is produced only for six training modes on (a) and for all available B-decays on (b).}~\label{fig:forest_channel}
\end{figure}


\begin{figure}
	\begin{center}
    	\begin{subfigure}[b]{0.45\textwidth}
    		\includegraphics[width=\textwidth]{../../img/roc_events.png} \caption{}
    	\end{subfigure} %forest_sel_top_train} 
    	\begin{subfigure}[b]{0.45\textwidth}
    		\includegraphics[width=\textwidth]{../../img/roc_events.png} \caption{} %forest_sel_top}
    	\end{subfigure}
    \end{center}
  \caption{Comparison between the original algorithm (\texttt{base}) and the random forest preselection approach: \texttt{xgb top 1 in channels}, \texttt{xgb top 2 in channels}. ROC curves is produced only for six training modes on (a) and for all available B-decays on (b).}~\label{fig:xgb_channel}
\end{figure}


\section{Speedup approaches with cleaned up samples}

\section{Conclusion}

\paragraph{Notes and Comments.}
The part of the topological trigger analysis connected with the machine learning studies and presented in this paper and \cite{run2_topo} are open source and can be found on the github (\url{https://github.com/tata-antares/LHCb-topo-trigger}).
%
% ---- Bibliography ----
%
\begin{thebibliography}{5}
%
\bibitem{run2_topo} Likhomanenko T., Ilten P., Khairullin E., Rogozhnikov A., Ustyuzhanin A., Williams M.: 
LHCb Topological Trigger Reoptimization. Journal of Physics: Conference Series 664 (2015) 082025 [arXiv:1510.00572]

\bibitem{detector}  A. A. Alves Jr. et al.: The LHCb Detector at the LHC, JINST 3, S08005 (2008)

\bibitem{run1_1} Aaij, R., {\em et al.} [LHCb Trigger Group]: 
The LHCb trigger and its performance.
JINST {\bf 8} P04022 (2013) [arXiv:1211.3055]

\bibitem{run1_2} Aaij R. et al.: 
Performance of the LHCb High Level Trigger in 2012, J. Phys., Conf. Ser. 513, 012001 (2014).

\bibitem{run1_topo} Gligorov V., Thomas C., Williams M.: 
The HLT inclusive B triggers. Technical Report
LHCb-PUB-2011-016. CERN-LHCb-PUB-2011-016. LHCb-INT-2011-030, CERN, Geneva, Sep 2011. LHCb-INT-2011-030.

\bibitem{bbdt} Gligorov V., Williams, M.:
Efficient, reliable and fast high-level triggering using a bonsai boosted decision tree.
JINST {\bf 8}, P02013 (2013) [arXiv:1210.6861]

\bibitem{forest} Breiman, L.: 
Random forests. Machine learning, 45(1), 5-32 (2001).

\bibitem{mn_paper} Gulin, A., Kuralenok, I., Pavlov, D.:
Winning the transfer learning track of Yahoo's Learning to Rank Challenge with YetiRank.
JMLR: Workshop and Conference Proceedings 14 (2011) 63 (Yandex MatrixNet).

% \bibitem {clar:eke}
% Clarke, F., Ekeland, I.:
% Nonlinear oscillations and
% boundary-value problems for Hamiltonian systems.
% Arch. Rat. Mech. Anal. 78, 315--333 (1982)

% \bibitem {clar:eke:2}
% Clarke, F., Ekeland, I.:
% Solutions p\'{e}riodiques, du
% p\'{e}riode donn\'{e}e, des \'{e}quations hamiltoniennes.
% Note CRAS Paris 287, 1013--1015 (1978)

% \bibitem {mich:tar}
% Michalek, R., Tarantello, G.:
% Subharmonic solutions with prescribed minimal
% period for nonautonomous Hamiltonian systems.
% J. Diff. Eq. 72, 28--55 (1988)

% \bibitem {tar}
% Tarantello, G.:
% Subharmonic solutions for Hamiltonian
% systems via a $\bbbz_{p}$ pseudoindex theory.
% Annali di Matematica Pura (to appear)

% \bibitem {rab}
% Rabinowitz, P.:
% On subharmonic solutions of a Hamiltonian system.
% Comm. Pure Appl. Math. 33, 609--633 (1980)

\end{thebibliography}

\end{document}
